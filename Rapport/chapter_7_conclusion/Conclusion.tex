\chapter{Conclusion}
Due to the short amount of time� for a project of this size, there are many possible improvements both on the implementations and as a consequence also on the simulations. In spite of this there are still many interesting tendencies that we have found through the simulations.

\\possible subjects for this section:\\
describe suggestions for a hybrid method. cherry picking advantages from our different approaches.\\
\\
backwards abstract. evaluate the thesis work and the results attained.\\
\\
describe possible future work on the algorithms.\\
\\
Identify the general good solving aspects. such as:\\
starting positions\\
easier to start with a solution, possibly a bad one, and improve. rather than two empty hands.\\
an promising approach is to find information on the enviroment that is not obviously provided. for instance reducing the amount of tiles. setting priority depending on the geometry. reading dynamical information out of previous itterations or current conditions. etc.\\

\section{Analysis}
An analysis of each algorithm,  evaluation of why it did or did not work. \par

Although the greedy algorithm was not fully implemented it was manually tested on several enviroments. There are two situations where the algorithm is known not to perform at its best. The first situation is when the enviroment is extremely symetrical. If this is the case the cost function will give equal values to several tiles and some pursuers will be given several equal alternatives. There is no handling for these events and the algorithm will probably freeze. The second situation is when two or more pursuers start at the same tile. This will result in that in every itteration they have the exact same conditions, thus they will make identical decissions and move as one single pursuer until their shared vision divides the enviroment into enough areas to be designated. When there are enough areas to be designated the pursuers will be forced to split up and go separate ways. This is also a problem due to symmetry in some sense.\\
\\