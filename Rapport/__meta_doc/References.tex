\addcontentsline{toc}{chapter}{References}
\begin{thebibliography}{99}
\selectlanguage{swedish}
 \bibitem{paper1} Johan Thunberg, Xiaoming Hu, Petter "Ogren, A Boolean Control Network Approach to Pursuit Evasion Problems in Polygonial Environments
\bibitem{paper2} Johan Thunberg, Petter "Ogren, A Mixed Integer Linear Programming approach to Pursuit Evasion Problems with optional Connectivity Constraints
\bibitem{paper3} Johan Thunberg, Petter "Ogren, An Iterative Mixed Integer Linear Programming Approach to Pursuit Evasion Problems in Polygonial Environments, 2010 IEEE International Conference on Robotics and Automation

\bibitem{NP} M. R. Garey, D. S. Johnson, Computers and Intractability: A Guide to the Theory of NP-Completeness

 \bibitem{C-bok}Brian W. Kernighan, Dennis M. Ritchie, "The C - Programming Language, (ANSI C Version)", Prentice-Hall of India Pvt. Ltd., New Delhi, 1998
%%%%% Fredrik referenser:
\bibitem{Dumitrescu08} A. Dumitrescu, I. Suzuki and P. Zylinski, Offline variants of the `Lion and Man' problem -- Some problems and techniques for measuring crowdedness and for safe path planning", Theoretical Computer Science, Vol. 399, June 2008, pp. 220-235.
\bibitem{GAHandbook1} Chambers, L.D. PRACTICAL HANDBOOK OF GENETIC ALGORITHMS (GAS) APPLICATIONS, VOL 1, CRC Press 1995, ISBN 0-8493-2519-6.
\bibitem{GAHandbook2} Chambers, L.D. GENETIC ALGORITHMS, VOL 2, CRC Press 1995, ISBN 0-8493-2529-3.
\bibitem{GA-ai} Genetic Algorithms WAREHOUSE, http://geneticalgorithms.ai-depot.com/Tutorial/Overview.html, 2011-05-01
\bibitem{quicksort} Microsoft Support, http://support.microsoft.com/kb/73853, 2011-05-02


%%%%% Felix referenser:
%%%%% Mats referenser:
\bibitem{online lecture} Online lecture, lecture 10 greedy algorithms part one, given by Prof.Sunder Vishwanathan from Department of Computer Science Engineering,IIT Bombay, site: http://nptel.iitm.ac.in/video.php?subjectId=106101060
\bibitem{introduction to Algorithms}Thomas H. Corman, Charles E. Leiserson, Ronald L. Rivest, Clifford Stein, "Introduction to Algorithms", 4th ed. MIT press 2001, chapter 16.
\bibitem{adk8}Lecture notes DD1352, Viggo Kann KTH, http://www.csc.kth.se/utbildning/kth/kurser/DD1352/adk10/schema/ADK-F8.pdf
\bibitem{hungarian}Lecture handouts, Beryl Castello, john hopkins university, http://www.ams.jhu.edu/~castello/362/Handouts/hungarian.pdf
%%%%%
%%%%% Exempel:
%%%%% \bibitem{key} Authors, Title, Publisher, Volume, Publish date, pages.
%%%%% key g�r att det g�r att referera genom \cite{key}, s�tt ett unikt v�rde p� key s� krockar vi inte. F�r d� (exempelvis) [1] i texten, vilket h�nvisar till referens 1.
%%%%%
%%%%%
%%%%%
%%%%%
%%%%%
%%%%% Gamla referenser fr�n exempel-rapporten:
%%%
%%%\bibitem{}Kenneth Hoffmann, Rey Kunze, " Linear Algebra", Prentice-Hall of India Pvt. Ltd., New Delhi, 1997
%%%\bibitem{}G.H. Golub and C. F. Van Loan , " Matrix Computations", Third Edition. The Johns Hopkins University Press, Baltimore, 1996
%%%\bibitem{}David A. Patterson, John L. Hennessy, "Computer Architecture, A Quantitative Approach", Morgan Kaufmann Publications Inc., San Mateo, California, USA, 1990
%%%\bibitem{}Jack Dongarra, Iain Duff, Danny Sorensen, and Henk van der Vorst, Numerical Linear Algebra for High-Performance Computing",Society for Industrial and Applied Mathematics, Philadelphia, 1998
%%%\bibitem{}Abraham Silberschatz, Peter Baer Galvin, "Operating System Concept", Addison Wesley, Reading Massachusetts, USA, 1998 
%%%\bibitem{}John P. Hayes, "Computer Architecture and Organization", McGraw-Hill International Company, Singapore, 1988 
%%%\bibitem{}PVM 3 User Guide and Reference Manual, Edited by Al Gist, Oak Ridge National Laboratory, Engineering Physics and Mathematics Divison, Mathematical Science Section, Oak Ridge, Tennessee, USA, 1991
%%% \bibitem{}PVM's HTTP Site, "http://www.epm.ornl.gov/pvm/"
\selectlanguage{english}
\end{thebibliography}
