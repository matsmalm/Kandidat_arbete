\chapter{Introduction}

\section{Objective}

The aim of this report is to construct heuristic algorithms that solve the pursuit and evasion problem in two dimensional polygonal environments. The original pursuit and evasion problem is fully formulated in chapter two. Informally it can be formulated as ``given an environment with static obstacles, and a specific number of pursuers, construct a search strategy for the group of pursuers such that the evader is guaranteed to be seen''. In \cite{paper3} a framework is presented for solving the problem. In this framework a method is provided for finding the optimal solution. In practice this algorithm is not very applicable though, since even for small areas the computational time is large. This motivates the introduction of heuristic methods. By sacrificing optimality, our aim is to use heuristic methods to find sufficiently good solutions within a reasonable computational time. These methods have, to the best of the authors' knowledge not been used in this context before.
\section{Background}

\subsection{What is optimization?}
Optimization is a mathematical discipline for solving various types of problems. The aim is to find the best available value of some objective function, given a defined domain.% Refer to other literature?
The mathematical theory behind optimization offers a variety of methods to solve a wide range of problems. To approach these problems and, find a solution, it is important to identify the characteristics of the problem considered. Relevant distinctions can be made by studying the problem's complexity. Complexity is strongly related to the computational time. There are many different complexity classes of problems, but two of the most fundamental are the P and NP.

subsection{P \& NP problems.}
%skriva om hela sektionen??
The distinction between these two reveals the difficulty of our problem. This will only be a brief description, for a more detailed and interesting description please see \cite{NP}.\\
Informaly one can say that P are problems that are easy, and NP are problems that are difficult. In NP ther is a subcalss called NP-complete.\\
NP-complete are the hardest problems in NP\cite{adk19}. Such a problem is NP-hard and in NP.
NP-hard are problems that are At least as hard as the hardest problems in NP. But such problems need not be in NP\cite{adk19}.\\
Related work has stated that the problem studied in this report are in fact of the class NP-hard \cite{paper1}.\\ % Is it paper one?
The conscvens of the problem beeing NP-hard is that one can not construct analytick algorithems that provieds an optimal solution in a resnabul amount of time.

\subsection{P \& NP problems.}
%skriva om hela sektionen??
The distinction between these two reveals the difficulty of our problem. This will only be a brief description, for a more detailed and interesting description please see \cite{NP}.

P is the set of problems which can be solved by a deterministic Turing machine using a polynomial amount of computation time.
Cobham Edmonds thesis holds that P is the class of computational problems which are "efficiently solvable" or "tractable". In practice, some problems not known to be in P have practical solutions, and some that are in P do not, but this is a useful rule of thumb.

NP refers to Non-deterministic Polynomial time, and can be intuitively described as the set of problems for which, a found solution can be verified to actually be the correct solution, using a deterministic Turing machine using a polynomial amount of computation time.

A set of descriptions that leads one to wonder whether
\begin{equation}
P = NP
\end{equation}

In what the essence is:
Suppose that solutions to a problem can be verified quickly. Then, can the solutions themselves also be computed quickly?
Or stated differently, does it exist such an algorithm that can calculate such a solution quickly?
An unsolved question considered by many to be the most important problem in the field of computer science. 

Within the class NP there are a few subclasses one of them are called NP-hard 
Related work has stated that the problem studied in this report are in fact of the class NP-hard \cite{paper1}, % Is it paper one?
and hence the chances of finding an algorithm that produces an optimal solution in polynomial time are greatly reduced. As stated above it is still an open question whether such algorithms actually exist for NP problems. 

\subsection{A near optimal solution.}
As mentioned in \cite{paper1} the problem under consideration is at least NP-hard. The consequences of this is that the chances of finding an optimal solution within reasonable computional time are seen to be very low. Thus we are imposed to sacrifice optimality, in order to gain computational effictivity. This sacrifice opens a large spectrum of possible approaches to the problem.


\subsection{Heuristic methods}
Heuristic methods are a branch of methods used in computer science and mathematics to make relevant guesses to find solutions to a problem. In general heuristic methods does not provide optimal solutions. Though sometimes, for specific situations, it can be proven that a certain heuristic algorithm's solutions are optimal. Also, there is no general way of proving whether a heuristic algorithm provides an optimal solution. Despite this fact heuristic methods often provide extremely efficient and relevant algorithms. Very often they reduce the computational time needed for a solution greatly. There are even several occasions where a heuristic method is prefered to an analytic method. A few examples are:
\begin{itemize}
\item When there is no known algorithm for solving a specific problem, a heuristic is the only way to approach a solution.
\item A algorithm can sometimes be difficult to implement. Heuristics can sometimes be used instead because they are easy to implement and known to produce good results.
\item When the problem is too difficult to solve efficiently and quickly with analytical methods. A heuristic method could overcome that and give an acceptable but probably not optimal solution.
\end{itemize}

The last point corresponds to the problem we are dealing with.