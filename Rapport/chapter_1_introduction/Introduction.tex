\chapter{Introduction}

\section{Objective}

The goal of this research is to solve pursuit \& evasion problems in a two dimensional environment, using heuristic problem solving methods.
The problem is fully explained in section 1.4.4. In a more relaxed manor it can be formulated as "given an environment with static obstacles and a specific number of pursuers, ensure that an evader cannot exist within the environment."

An algorithm providing an optimal solution already exists.
The problem yet to be solved is the problem of finding a solution within a reasonable computational time. This is here where heuristic methods could provide interesting results.

With risk for loss of optimality, our goal is to use heuristic methods to find a solution within a reasonable computational time.
 
\section{Organization of the report}
\begin{description}
\item Chapter 2 contains a detailed formulation of the problem and are approach to it.
\item Chapter 3 explains the simulation environment we built.
\item Chapter 4 contains a description of the algorithms we decided to use.
\item Chapter 5 contains the results of our simulations.
\item Chapter 6 contains a discussion of our results and a comparison.
\item Chapter 7 contains conclusions of our work.
\end{description}

\section{Related work}

Our work is based on the work of papers \cite{paper1}-\cite{paper3}.

\section{Background}

\subsection{What is optimization}

Optimization is a mathematical discipline for solving various types of problems. The aim is to find the best available value of some objective function, given a defined domain.
% Refer to other literature?

Optimization theory offers a variety of solution methods to a wide range of problems. To approach theses problems and find a solution, it is important to identify what kind of problem one deals with.
Relevant distinctions can be made by studying the problem's complexity.

Complexity is strongly related to the computational time. There are many different complexity classes of problems, but two of the most fundamental are the P and NP.

\subsection{P \& NP problems}

The distinction between these two reveals the difficulty of our problem. This will only be a brief description, for a more detailed and interesting description please see \cite{NP}.

P is the set of problems which can be solved by a deterministic Turing machine using a polynomial amount of computation time.
Cobham Edmonds thesis holds that P is the class of computational problems which are "efficiently solvable" or "tractable". In practice, some problems not known to be in P have practical solutions, and some that are in P do not, but this is a useful rule of thumb.

NP refers to Non-deterministic Polynomial time, and can be intuitively described as the set of problems for which, a found solution can be verified to actually be the correct solution, using a deterministic Turing machine using a polynomial amount of computation time.

A set of descriptions that leads one to wonder whether
\begin{equation}
P = NP
\end{equation}

In what the essence is:
Suppose that solutions to a problem can be verified quickly. Then, can the solutions themselves also be computed quickly?
Or stated differently, does it exist such an algorithm that can calculate such a solution quickly?
An unsolved question considered by many to be the most important problem in the field of computer science. 

Within the class NP there are a few subclasses one of them are called NP-hard 
Related work has stated that the problem studied in this report are in fact of the class NP-hard \cite{paper1}, % Is it paper one?
and hence the chances of finding an algorithm that produces an optimal solution in polynomial time are greatly reduced. As stated above it is still an open question whether such algorithms actually exist for NP problems. 

\subsection{(the need for) A near optimal solution}

Due to the fact that the problem in question is NP-hard and thus the possibilities of finding an optimal solution within reasonable computing time are/is seen to be very low. The logical thread has brought us to be prepared to sacrifice optimality.
This sacrifice leads us to a larger spectrum of possible approaches to the problem.


 \subsection {Pursuit and evasion problem with multiple pursuers.}
%The pursuit evasion problem, also known as "Lion and man" \cite{Dumitrescu08}, is a problem where an pursuer is to find an evader in an environment. An extension to this is the multi pursuer version, which is used in \cite{paper1},\cite{paper2} and \cite{paper3}.\\

Following the previous work of Johan Thunberg \cite{paper1}, the pursuers and evader are modeled as points moving in the polygonal free space, F . Let  $e(\tau )$ denote the position of the evader at time $\tau \geq 0$. It is assumed that $e : \lbrack 0, \infty) \to F$ is continuous, and that the evader is able to move arbitrarily fast. The initial position e(0) and path e is not known to the pursuers. At each time instant, F is partitioned into two subsets, the cleared and the contaminated, where the latter might contain the evader and the former might not. Given N pursuers, let $p_i (\tau ) : \lbrack 0, \infty) \to F$ denote the position of the i:th pursuer, and $P = \lbrace p_1 , . . . , p_N \rbrace$ be the motion strategy of the whole group of pursuers. Let V (q) denote the set of all points that are visible from $q \subset F$ , i.e., the line segment joining q and any point in V (q) is contained in F .\\
\\
\textbf{The original Problem (Pursuit Evasion).} \emph{ Given an evader, a set of N pursuers and a polygonal free space F , �nd a solution strategy P such that for every continuos function $e : \lbrack 0, \infty) \to F$ there exists a time $\tau$ and an i such that $e(\tau ) \subset V (p_i (\tau ))$, i.e., the pursuer will always be seen by some evader, regardless of its path. }


\subsection{Heuristic methods}

Heuristic methods are a branch of methods used in computer science and mathematics to make educated guesses to help find solutions to a problem. Generally heuristic methods does not provide optimal solutions but it offers a reasonably fast computed solution. Heuristic methods are therefore often used, and very powerful in combination with deterministic methods \footnote{deterministic method  or deterministic algorithms are algorithms that Given a particular input always produce the same output.} to speed up the process.

The characteristics of a heuristic is partly that there is no evidence of that it works, or else it would be a deterministic method. The idea of heuristic methods is thus to improve the chances of finding the desired solution. Pure random guesses forms no heuristic method.

There are several occasions where a heuristic method is preferable to a deterministic method.
Here are a few:

\begin{itemize}
\item There is no known algorithm for solving a specific problem, a heuristic is the only way to approach a solution
\item A good algorithm can be difficult to implement and heuristics can sometimes be accepted because it is easy to describe in a programming language and that heuristic methods are known to produce good results.
\item The problem is too difficult to solve efficiently and quickly with deterministic methods. A heuristic method could overcome that and give an acceptable but probably not an optimal solution
\end{itemize}

The last point corresponds to the problem we are dealing with.