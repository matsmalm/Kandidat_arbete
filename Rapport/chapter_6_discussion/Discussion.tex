\chapter{Discussion}
This chapter contains analysis of the data collected when running the algorithms and conclusions made during the working process.

\section{The simulation}
The simulation environment developed was usable, but is not without limitations some advantages and disadvantages are listed below.\\\\
The simulation environment can not merge tiles into one node. Since the complexity of many of the algorithms that were used depends on the number of nodes, merging nodes should lead to a decreased computation time. As the memory usage of the graph is proportional to the number of tiles, memory usage would also decrease.\\\\
Using a graph means that existing algorithms that are already written for graphs %such as?
can be used, which facilitates implementation. During implementation it is also possible for the nodes to contain a collection information.\\
The decision to read environments from files makes the simulation environment usable for many different environments. New environments could easily either be generated or added by hand.
%utv�rdera f�rdelar och nackdelar som simuleringsmilj�n besitter.
	%f�rklara visions tillkortakommande
	%kan inte mergea, kr�ver mycker minne
	%graph-approach var bra ide?
	%applicerbar pga filskrivning
\section{Analysis of the simulations}
First off it should be noted that when varying the size of the enviroment certain parameters of the algorithms had to be adjusted. For enviroments of intermediate size a correct adjustment of parameters yielded solutions of significantly higher quality. For larger enviroments it was necessary in order to attain any solution at all. We found no analytical way of adjusting the parameters. A good understanding of the implementation in combination with intuition and testing usually solved the issue. This does not exclude the possibility that there might be a more descisive approach. In fact, based mainly on our intuition, it is most probable.\\
\\As seen in the table \ref{SimData}, given in the previous chapter, for small enviroments both algorithms efficiently provides solutions of high quality, even with a small amount of pursuers. The optimality of the solutions on the randomly generated enviroments is not proven, but it's likely that the provided solutions in most cases are optimal. This assumtion is motivated by the fact that the quality of the solutions rarely differs and the path length is typicaly only a few steps long. It should be noted that the average computational time for these solutions is 3 milliseconds for the tabu search algorithm and 37 milliseconds for the genetic algorithm. These are highly efficient results compared to related work \cite{paper1}.\\%n�mn och j�mf�r manhattan resultat
\\When the size of the enviroment is increased the solutions found are less probable to be optimal. This assumption is motivated by the fact that the quality of the solutions now vary strongly. Also the computational time severely increases with the size of the enviroment. An interesting observation though is to compare how the average computational time for the two algorithms changes as the difficulty increases. The tabu search algorithm is highly efficient if it is able to find a first complete solution. But when running on the bigger areas the average time diverges. On the other hand the genetic algorithm provides reasonable average computational times, and is more reliable to find some solution even for difficult enviroments.\\
\\By combining an indepth understanding of the implementation with the data aquired the following conclusions, and fundamental differences, can be made on the implementation of the algorithms. The tabu algortithm is strongly dependent on finding a complete solution of sufficient quality, in order to be efficient. The reason behind this is that a maximum step length is set dynamicaly by the complete solution found, and tiles not visited in this solution are rendered tabu. Thus, with a complete solution of bad quality there are still a vast amount of aternative paths to consider, which in turn affects the efficency.\\
\\The genetic algorithm does not have the same dependency of finding a complete solution, due to its reproduction properties. This motivates the more robust results on efficency for larger areas, compared to the tabu algorithm. 		
\section{Starting positions}
%l�sningskvalitet starkt beroende av startpositioner.
	%formulera tydligt vilka startf�rh�llanden som avses i problemst�llning hos framtida arbeten.
	%vi har slumpstart, varf�r? vad kan vi se?
	%bed�m kvalitet p� given l�sning utifr�n omst�ndigheter.
	
One of our intentions with this report was to compare 

\section{When do the algorithms fail?}	
%n�r skiter det sig och varf�r?

\section{Future work}
%future work:
	%f�rb�ttra v�rat arbete programmeringsm�ssigt
	%implementera ideer uteblivna pga tidsbrist
		%uteblivna tabuvillkor
		%f�rb�ttra simuleringsmilj�n
		%hybridmetoder
		%implementera greedy, motviera varf�r.
		%om man kunde visa att en optimal l�sning kan formuleras som superposition av dell�sningar skulle algorithmerna kunna �ndras i grunden och bli extremty mycket effektivare.
Improvement on the implementations of the algorithms and the simulation enviroment is probable to give more reliable and promising results. Due to the short amount of time given for a project of this size many short cuts were made intentionally. Another interesting suggestion is to combine the algorithms and create a hybrid method. \\
\\Since the algorithms constructed in this paper are considered to be very efficient for smaller problems it is suggested to investigate whether an global optimal solution could be attained by solving local subproblems.







		
