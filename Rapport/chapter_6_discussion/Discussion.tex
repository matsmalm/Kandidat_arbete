\chapter{Discussion}
This chapter contains analysis of the data collected when running the algorithms and conclusions made during the working process.

\section{The simulation}
The simulation environment developed was usable, but is not without limitations some advantages and disadvantages are listed below.\\\\
The simulation environment can not merge tiles into one node. Since the complexity of many of the algorithms that were used depends on the number of nodes, merging nodes should lead to a decreased computation time. As the memory usage of the graph is proportional to the number of tiles, memory usage would also decrease.\\\\
Using a graph means that existing algorithms that are already written for graphs %such as?
can be used, which facilitates implementation. During implementation it is also possible for the nodes to contain a collection information.\\
The decision to read environments from files makes the simulation environment usable for many different environments. New environments could easily either be generated or added by hand.
%utv�rdera f�rdelar och nackdelar som simuleringsmilj�n besitter.
	%f�rklara visions tillkortakommande
	%kan inte mergea, kr�ver mycker minne
	%graph-approach var bra ide?
	%applicerbar pga filskrivning
\section{Analysis of the simulations}
First off it should be noted that when varying the size of the environment certain parameters of the algorithms had to be adjusted. For environments of intermediate size a correct adjustment of parameters yielded solutions of significantly higher quality. For larger environments it was necessary in order to attain any solution at all. We found no analytical way of adjusting the parameters. A good understanding of the implementation in combination with intuition and testing usually solved the issue. This does not exclude the possibility that there might be a more decisive approach. In fact, based mainly on our intuition, it is most probable.\\
\\As seen in the table \ref{SimData}, given in the previous chapter, for small environments both algorithms efficiently provides solutions of high quality, even with a small amount of pursuers. The optimality of the solutions on the randomly generated environments is not proven, but it's likely that the provided solutions in most cases are optimal. This assumption is motivated by the fact that the quality of the solutions rarely differs and the path length is typically only a few steps long. It should be noted that the average computational time for these solutions is 3 milliseconds for the tabu search algorithm and 37 milliseconds for the genetic algorithm. These are highly efficient results compared to related work \cite{paper1}.\\%n�mn och j�mf�r manhattan resultat
\\When the size of the enviroment is increased the solutions found are less probable to be optimal. This assumption is motivated by the fact that the quality of the solutions now vary strongly. Also the computational time severely increases with the size of the environment. An interesting observation though is to compare how the average computational time for the two algorithms changes as the difficulty increases. The tabu search algorithm is highly efficient if it is able to find a first complete solution. But when running on the bigger areas the average time diverges. On the other hand the genetic algorithm provides reasonable average computational times, and is more reliable to find some solution even for difficult environments.\\
\\By combining an in-depth understanding of the implementation with the data acquired the following conclusions, and fundamental differences, can be made on the implementation of the algorithms. The tabu algorithm is strongly dependent on finding a complete solution of sufficient quality, in order to be efficient. The reason behind this is that a maximum step length is set dynamically by the complete solution found, and tiles not visited in this solution are rendered tabu. Thus, with a complete solution of bad quality there are still a vast amount of alternative paths to consider, which in turn affects the efficiency.\\
\\The genetic algorithm does not have the same dependency of finding a complete solution, due to its reproduction properties. This motivates the more robust results on efficiency for larger areas, compared to the tabu algorithm. 
%%		
%\section{Starting positions}
%l�sningskvalitet starkt beroende av startpositioner.
	%formulera tydligt vilka startf�rh�llanden som avses i problemst�llning hos framtida arbeten.
	%vi har slumpstart, varf�r? vad kan vi se?
	%bed�m kvalitet p� given l�sning utifr�n omst�ndigheter.
	
%One of our intentions with this report was to compare 
\section{Analysis}
%n�r skiter det sig och varf�r? Samt analys av enskilda algoritmerna.
A short analysis of each algorithm,  evaluation of why it did or did not work.

\subsection{Genetic}
The genetic algorithm ran a vast number of times for different environments, and found a solution for most environments that it was given. The quality of the solutions did however vary, as can be seen in Table \ref{SimData}. The reason for the variation in number of required steps compared to the tabu algorithm is most probably premature convergence. As the crossover operator implemented was simple, with only swapping paths of the pursuers, it was not possible to explore a vast number of feasible solutions. An attempt to decrease the effect of this was made by the increase in rate of mutations.\\
As the algorithm does not consider the environment, it should not have problem with the shape of the environment. With the implementation of crossover that is used, the algorithm is required to have a high rate of mutations. A large number of pursuers should however be more advantageous for the algorithm, as there are more ways to combine paths of the pursuers.
\subsection{Greedy}
Although the greedy algorithm was not fully implemented it was manually tested on several environments. There are two situations where the algorithm is known not to perform at its best. The first situation is when the environment is extremely symmetrical. If this is the case the cost function will give equal values to several tiles and some pursuers will be given several equal alternatives. There is no handling for these events and the algorithm will probably freeze. The second situation is when two or more pursuers start at the same tile. This will result in that in every iteration they have the exact same conditions, thus they will make identical decisions and move as one single pursuer until their shared vision divides the environment into enough areas to be designated. When there are enough areas to be designated the pursuers will be forced to split up and go separate ways. This is also a problem due to symmetry in some sense.

\subsection{Tabu}
The tabu algorithm went through many simulations, and performed relatively well. The most evident strength of the algorithm is tabu rule (5). Which effectively reduced the search space when a complete solution is found. This fact also highlights the main weakness of the algorithm, namely that the algorithm is strongly dependent on finding the first complete solution in order to be efficient. Under certain conditions the tabu rules them selves are known to hinder the search for a complete solution. An example of this is when the starting positions are very spread out.  If the pursuers are spread out they must meet in order to be able to cooperate. Since the tabu rule (1) prohibits the pursuers from returning to a tile previously visited the path taken by the pursuers, in order to meet, hinders further cooperative search in these directions. This problem could probably have been avoided by adjusting the parameter save\_past\_X\_number\_of\_steps. However this was not investigated in detail. There is also a known problem with tabu rule (1) for a certain type of geometry. Suppose that there is closed area with only one entrance, and that this entrance is a corridor. Once a pursuer enters a geometry of this type the tabu rule (1) will hinder the pursuer to exit the area. This in turn inhibits the general cooperation of the pursuer team. This could be one of the reasons why the tabu algorithm sometimes failed in find a solutions for a small number of pursuers.

%% Identify good solving aspects
\subsection{Aspects for identifying good solutions}
Identify the general good solving aspects. such as:
\begin{itemize}
\item Starting positions
\item Easier to start with a solution, possibly a bad one, and improve. rather than two empty hands.
\item An promising approach is to find information on the environment that is not obviously provided. for instance reducing the amount of tiles. setting priority depending on the geometry. reading dynamical information out of previous iterations or current conditions. etc.
\end{itemize}

\section{Future work}
%future work:
	%f�rb�ttra v�rat arbete programmeringsm�ssigt
	%implementera ideer uteblivna pga tidsbrist
		%uteblivna tabuvillkor
		%f�rb�ttra simuleringsmilj�n
		%hybridmetoder
		%implementera greedy, motviera varf�r.
		%om man kunde visa att en optimal l�sning kan formuleras som superposition av dell�sningar skulle algorithmerna kunna �ndras i grunden och bli extremty mycket effektivare.
Improvement on the implementations of the algorithms and the simulation environment is probable to give more reliable and promising results. Due to the short amount of time given for a project of this size many short cuts were made intentionally. Another interesting suggestion is to combine the algorithms and create a hybrid method. \\
\\Since the algorithms constructed in this paper are considered to be very efficient for smaller problems it is suggested to investigate whether an global optimal solution could be attained by solving local sub-problems.
