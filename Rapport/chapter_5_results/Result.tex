\chapter{Results}
The algorithms were run on iMac's with 2.66GHz Intel Core2Duo and 4GB of RAM. Each machine was given environments of a specified size\footnote{Environments of size 5x5, 9x9, 10x10, 15x15 and 20x20 were used}, where the density of obstacles was either 25\% or 40\%. Figure \ref{Envs} shows examples of environments of size 10x10 with (a) 25\% obstacle density and (b) 40\% density that were used.\\
\\For the random enviroment simulations each machine was given 10 randomly generated enviroments of equal size. Five starting positions were generated, twice for each enviroment. To decrease the amount of pursuers the starting positions lastly generated were omitted. Each setup of starting positions, number of pursuers and enviroment was run with the same conditions four times. Each algorithm was given a set of parameters that were tuned in order to find a solution with two pursuers, however each environment was run with 5 to 2 pursuers. The algorithms were also given a maximum number of steps allowed for each pursuer to take, set to be long enough to find a solution.The algorithms were then to find a solution of as few steps as possible for a given number of pursuers. If for a given amount of pursuers neither of the algorithms were able to find a solution within the maximum steps allowed, no further attempts of finding a solution with these conditions were made. Neither were any furhter attempts made for fewer pursuers on these conditions. In table \ref{SimData} the data acquired is summarized.\\
\\For the Manhattan\footnote{A special symmetric environment, see \cite{paper1}} grid the simulation conditions were different. A 5x5 and a 9x9 Manhattan grid was used with 5 to 2 pursuers, starting positions was the upper left corner for all pursuers and cases.\\
%fyll ut med hur!!!

%\\In table \ref{SimData} the "Average step difference" is given by the difference in steps used for the genetic algorithm and tabu search algorithm in successful cases, divided by the number of successful runs for the given environment and number of pursuers. The "Average time" is the time for each run divided by the number of runs, where both successful and unsuccessful runs were counted.
%% Tables with data
In table \ref{SimData} the "Average step difference" describes the difference in path length of complete solutions. This value is attained by subtracting the sum of all complete path lengths of the tabu algorithm from the sum of all complete path lengths of the genetic algorithm and divide this by the number of successfull runs for the given enviroment. The "Average time" is the time for each run divided by the number of runs, where both successful and unsuccessful runs were counted.

\begin{center}
\begin{table}[t!hb]
\noindent\makebox[\textwidth]{%
\begin{tabular}{| c | r | r | r | r | r | r | r | r | }
\hline
%%%%%%%%%%%%%%%%%%%%%%%%%%%%%%%%%%%%%%%%%%%%%%%%%%%%%%%%%%%%%%%%%%%%%%
%%                                                                  %%
%%  This is a LaTeX2e table fragment exported from Gnumeric.        %%
%%                                                                  %%
%%%%%%%%%%%%%%%%%%%%%%%%%%%%%%%%%%%%%%%%%%%%%%%%%%%%%%%%%%%%%%%%%%%%%%
Size	&Pursuers	&Runs	&Unsolved	&Unsolved	&Average step difference	&Avgerage time	&Average time \\
	&	&	&(Genetic)	&(Tabu)	&(Genetic - tabu)	&Genetic (sec)	&Tabu (sec)\\
\hline
5x5	&5	&80	&0	&0	&0.00	&0.037	&0.003\\
5x5	&4	&80	&0	&0	&0.00	&0.049	&0.008\\
5x5	&3	&80	&0	&0	&0.01	&0.084	&0.010\\
5x5	&2	&80	&0	&0	&0.00	&0.116	&0.012\\
	&	&	&	&	&	&	&\\
10x10	&5	&96	&0	&0	&2.38	&45.653	&18.046\\
10x10	&4	&96	&0	&0	&7.28	&128.613	&44.249\\
10x10	&3	&95	&1	&12	&16.83	&312.450	&76.693\\
10x10	&2	&67	&14	&22	&35.11	&537.163	&80.843\\
	&	&	&	&	&	&	&\\
15x15	&5	&42	&10	&23	&54.65	&560.978	&1676.596\\
15x15	&4	&17	&7	&8	&68.43	&663.767	&1278.498\\
15x15	&3	&3	&3	&3	&0.00	&1004.250	&459.086\\
	&	&	&	&	&	&	&\\
20x20	&5	&4	&4	&4	&0.00	&9521.394	&8959.600\\
	&	&	&	&	&	&	&\\
Mhtn (5x5)	&5	&16	&0	&0	&-0.62	&2.722	&0.019\\
Mhtn (5x5)	&4	&16	&0	&0	&-0.12	&3.071	&0.021\\
Mhtn (5x5)	&3	&16	&0	&0	&-0.75	&3.118	&0.019\\
Mhtn (5x5)	&2	&16	&1	&0	&-0.53	&2.998	&0.021\\
	&	&	&	&	&	&	&\\
Mhtn (9x9)	&5	&24	&0	&0	&0.42	&23.161	&21.023\\
Mhtn (9x9)	&4	&24	&0	&0	&0.08	&22.654	&12.883\\
Mhtn (9x9)	&3	&24	&0	&0	&-0.58	&23.102	&22.920\\
Mhtn (9x9)	&2	&24	&0	&0	&1.33	&21.221	&16.246\\
 
\hline
\end{tabular} }
\caption{Data from simulations}
\label{SimData}
\end{table}
\end{center}
%% 
%% Environments
\begin{figure}[t!hb]
\begin{center}
\begin{tabular}{| p{0.1cm} | p{0.1cm} | p{0.1cm} | p{0.1cm} | p{0.1cm} | p{0.1cm} | p{0.1cm} | p{0.1cm} | p{0.1cm} | p{0.1cm} | }
\hline
%%%%%%%%%%%%%%%%%%%%%%%%%%%%%%%%%%%%%%%%%%%%%%%%%%%%%%%%%%%%%%%%%%%%%%
%%                                                                  %%
%%  This is a LaTeX2e table fragment exported from Gnumeric.        %%
%%                                                                  %%
%%%%%%%%%%%%%%%%%%%%%%%%%%%%%%%%%%%%%%%%%%%%%%%%%%%%%%%%%%%%%%%%%%%%%%
	&	&0\cellcolor{black}	&	&	&	&	&0\cellcolor{black}	&	&0\cellcolor{black}\\
\hline
	&	&	&	&	&0\cellcolor{black}	&	&	&	&0\cellcolor{black}\\
\hline
0\cellcolor{black}	&0\cellcolor{black}	&	&	&	&	&0\cellcolor{black}	&	&0\cellcolor{black}	&\\
\hline
	&	&	&	&	&0\cellcolor{black}	&	&	&	&\\
\hline
	&	&	&	&	&	&0\cellcolor{black}	&	&0\cellcolor{black}	&0\cellcolor{black}\\
\hline
0\cellcolor{black}	&	&	&	&	&	&	&	&0\cellcolor{black}	&0\cellcolor{black}\\
\hline
0\cellcolor{black}	&0\cellcolor{black}	&	&0\cellcolor{black}	&	&	&	&0\cellcolor{black}	&	&\\
\hline
	&	&	&	&	&0\cellcolor{black}	&	&	&	&\\
\hline
0\cellcolor{black}	&	&	&	&	&	&	&	&	&\\
\hline
	&	&	&	&	&0\cellcolor{black}	&0\cellcolor{black}	&0\cellcolor{black}	&	&\\
 
\hline
\end{tabular}
\hspace{0.5cm}
\begin{tabular}{| p{0.1cm} | p{0.1cm} | p{0.1cm} | p{0.1cm} | p{0.1cm} | p{0.1cm} | p{0.1cm} | p{0.1cm} | p{0.1cm} | p{0.1cm} | }
\hline
\input{chapter_5_results/Environments-10x10-25-2.tex} 
\hline
\end{tabular}
\vspace{0.1cm}\\(a) 10x10 environments with 25\% obstacle density\\*
\end{center}
\vspace{0.05cm}
\begin{center}
\begin{tabular}{| p{0.1cm} | p{0.1cm} | p{0.1cm} | p{0.1cm} | p{0.1cm} | p{0.1cm} | p{0.1cm} | p{0.1cm} | p{0.1cm} | p{0.1cm} | }
\hline
\input{chapter_5_results/Environments-10x10-40-1.tex} 
\hline
\end{tabular}
\hspace{0.5cm}
\begin{tabular}{| p{0.1cm} | p{0.1cm} | p{0.1cm} | p{0.1cm} | p{0.1cm} | p{0.1cm} | p{0.1cm} | p{0.1cm} | p{0.1cm} | p{0.1cm} | }
\hline
\input{chapter_5_results/Environments-10x10-40-2.tex} 
\hline
\end{tabular}
\vspace{0.1cm}\\(a) 10x10 environments with 40\% obstacle density\\*
\caption{Examples of environments used}
\end{center}
\label{Envs}
\end{figure}
%% End of environments
%% End of tables
\newpage From the simulations the following observations were made:
\begin{itemize}
\item{For environments of size 15x15 and larger the amount of RAM memory used was significant.}
\item{The results were dependent on parameter values.}
\item{For environments of size 5x5 were solved with good quality for both the tabu search and the genetic algorithm, however tabu search were faster in finding a solution.}
\item{Environments of size 10x10 and larger were not always solved with less than 4 pursuers.}
\item{When tabu search found a solution, it was often of better quality than the genetic algorithm.}
\item{For environments of size 10x10 and larger, the genetic algorithm found a solution more often than tabu search.}
\item{For environments of size 10x10 and larger, the quality of the solution, if found, varied in steps and computation time.}
\end{itemize}

%%Statistics, tables and a description of the tables. Also motivation to why we have chosen these tables etc.\\\\ %%Results for MILP\footnote{Mixed Integer Linear Programming, see \cite{paper3}} evaluation could also be added here.\\\\